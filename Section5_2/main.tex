\documentclass{tufte-handout}

\title{Section 5.2: Trigonometric Functions of Real Numbers}

\author[AW]{Ammon Washburn}

\usepackage{graphicx} % allow embedded images
  \setkeys{Gin}{width=\linewidth,totalheight=\textheight,keepaspectratio}
  \graphicspath{{graphics/}} % set of paths to search for images
\usepackage{amsmath}  % extended mathematics
\usepackage{booktabs} % book-quality tables
\usepackage{units}    % non-stacked fractions and better unit spacing
\usepackage{multicol} % multiple column layout facilities
\usepackage{lipsum}   % filler text
\usepackage{enumerate}
\usepackage{wrapfig}
\usepackage{fancyvrb} % extended verbatim environments
  \fvset{fontsize=\normalsize}% default font size for fancy-verbatim environments
\usepackage{tikz}
\usepackage{subcaption}
\captionsetup{compatibility=false}
\usepackage{mathtools}
\usepackage{graphicx}
\usepackage{amssymb}
\usepackage{enumerate}
\usepackage{color}
\usepackage{fancyvrb}
\usepackage{breqn}
\usepackage{fancyhdr}
\usepackage{multicol}
%\usepackage[latin1]{inputenc}
\usepackage{tikz}
\usepackage{pgfplots}
\pgfplotsset{compat=1.8}

\definecolor{dkgreen}{rgb}{0,0.6,0}
\definecolor{gray}{rgb}{0.5,0.5,0.5}
\definecolor{mauve}{rgb}{0.58,0,0.82}

\newcommand{\R}[1]{\mathbb{R}^{#1}}

\pgfplotsset{vasymptote/.style={
    before end axis/.append code={
        \draw[densely dashed] ({rel axis cs:0,0} -| {axis cs:#1,0})
        -- ({rel axis cs:0,1} -| {axis cs:#1,0});
    }
}}
\pgfplotsset{hasymptote/.style={
    before end axis/.append code={
    	%\draw (axis cs:0,1) -- ({axis cs:0,1}-|{rel axis cs:1,0});
        \draw[densely dashed] ({rel axis cs:0,1} -| {axis cs:0,#1})
        -- ({rel axis cs:0,0} -| {axis cs:0,#1});
    }
}}

% Standardize command font styles and environments
\newcommand{\doccmd}[1]{\texttt{\textbackslash#1}}% command name -- adds backslash automatically
\newcommand{\docopt}[1]{\ensuremath{\langle}\textrm{\textit{#1}}\ensuremath{\rangle}}% optional command argument
\newcommand{\docarg}[1]{\textrm{\textit{#1}}}% (required) command argument
\newcommand{\docenv}[1]{\textsf{#1}}% environment name
\newcommand{\docpkg}[1]{\texttt{#1}}% package name
\newcommand{\doccls}[1]{\texttt{#1}}% document class name
\newcommand{\docclsopt}[1]{\texttt{#1}}% document class option name
\newenvironment{docspec}{\begin{quote}\noindent}{\end{quote}}% command specification environment
\newcommand{\Z}[1]{\mathbb{Z}^{#1}}

\newtheorem{mydef}{Definition}
\providecommand{\floor}[1]{\left \lfloor #1 \right \rfloor }


\begin{document}


\section{The Trigonometric Functions}
Let $t \in \R{}$, and $P(x, y)$ be the terminal point on the unit circle determined by $t$. 
Define
\begin{enumerate}
\item $\sin(t) = y$ {\color{blue} sine}
\item $\cos(t) = x$ {\color{blue} cosine}
\item $tan(t) = \frac{y}{x} {\color{green} = \frac{\sin(t)}{\cos(t)}}$ when $x \not= 0$ {\color{blue} tangent} 
\item $\csc(t) = \frac{1}{y} {\color{green} = \frac{1}{\sin(t)}}$ when $y \not= 0$ {\color{blue} cosecant}
\item $\sec(t) = \frac{1}{x} {\color{green} = \frac{1}{\cos(t)}}$ when $x \not= 0$ {\color{blue} secant}
\item $\cot(t) = \frac{x}{y} {\color{green} = \frac{\cos(t)}{\sin(t)}}$ when $y \not= 0$ {\color{blue} cotangent}
\end{enumerate}

\subsection{Table for sines and consines}
\begin{tabular}{c|c|c}
$t$ & $\sin(t)$ & $\cos(t)$ \\ \hline
$0$     & $\sqrt{0}/2$ & $\sqrt{4}/2$ \\
$\pi/6$ & $\sqrt{1}/2$ & $\sqrt{3}/2$ \\
$\pi/4$ & $\sqrt{2}/2$ & $\sqrt{2}/2$ \\
$\pi/3$ & $\sqrt{3}/2$ & $\sqrt{1}/2$ \\
$\pi/2$ & $\sqrt{4}/2$ & $\sqrt{0}/2$
\end{tabular}

\subsection{Examples}
\begin{enumerate}
\item $t = \pi/3$
\item $t = \pi/2$
\end{enumerate}
{\color{blue} Solutions:
\begin{tabular}{c|c|c|c|c|c|c}
$t$ & $\sin(t)$ & $\cos(t)$ & $\tan(t)$ & $\csc(t)$ & $\sec(t)$ & $\cot(t)$ \\ \hline
$\pi/3$ & $\sqrt{3}/2$ & 1/2 & $\sqrt{3}$ & $2\sqrt{3}/3$ & $2$ & $\sqrt{3}/3$ \\
$\pi/2$ & 1 & 0 & $-$ & 1 & $-$ & 0
\end{tabular}
}

\section{Domains of Trigonometric Functions}
\begin{enumerate}
\item $\sin, \cos$: $\R{}$
\item $\tan, \sec$: $\R{}$ except $\frac{\pi}{2} + n\pi$ for $n \in \Z{}$
\item $\cot, \csc$: $\R{}$ except $n\pi$ for $n \in \Z{}$
\end{enumerate}

\section{Values of the Trigonometric Functions}

\subsection{Signs of Trig Functions}
\begin{tabular}{c|c|c}
Quadrant & Positive Functions & Negative Functions \\ \hline
I & all & none \\
II & sin, csc & rest \\
III & tan, cot & rest \\
IV & cos, sec & rest
\end{tabular} \qquad
Mnemonic: 
\begin{tabular}{c|c}
Students & All \\ \hline
Take & Calc
\end{tabular}

\subsection{Examples}
\begin{enumerate}[(a)]
\item $\cos\left( \frac{2\pi}{3} \right)$
\item $\tan\left( \frac{-\pi}{3} \right)$
\item $\sin\left( \frac{19\pi}{4} \right)$
\end{enumerate}
{\color{blue}
\begin{enumerate}[(a)]
\item Reference number is $\bar{t} = \pi/3$; in QII, so 
\begin{align*}
\cos\left( \frac{2\pi}{3} \right) = -\cos\left( \frac{\pi}{3} \right) = -\frac{1}{2}
\end{align*}
\item Reference number is $\bar{t} = \pi/3$, in QIV, so
\begin{align*}
\tan\left( \frac{-\pi}{3} \right) = -\tan\left( \frac{\pi}{3} \right) = -\sqrt{3}
\end{align*}
\item Reference number is $\bar{t} = \pi/4$, in QII, so
\begin{align*}
\sin\left( \frac{19\pi}{4} \right) = \sin\left( \frac{3\pi}{4} \right) = \frac{\sqrt{2}}{2}
\end{align*}
\end{enumerate}
}

\subsection{Note for Calculators}
{\color{red} Make sure calculator is set to radian mode!}
MODE $>$ Radian (3rd down) 

\subsection{Even and Odd Properties}
\begin{enumerate}
\item Sine, cosecant, tangent, and cotangent are all \textbf{odd} functions
\item Cosine and secant are both \textbf{even} functions
\end{enumerate}
\begin{align*}
&\sin(-t) = -\sin(t) &\cos(-t) = \cos(t) &\qquad\tan(-t) = -\tan(t) \\
&\csc(-t) = -\csc(t) &\sec(-t) = \sec(t) &\qquad\cot(-t) = -\cot(t) 
\end{align*}

\subsection{Examples}
\begin{enumerate}[(a)]
\item $\sin(-\pi/6) {\color{blue} = -\sin(\pi/6) = -1/2}$ \\
\item $\cos(-\pi/4) {\color{blue} = \cos(\pi/4) = \sqrt{2}/2}$
\end{enumerate}

\section{Fundamental Identities}
\begin{enumerate}
\item Reciprocal Identities:
\begin{align*}
\csc(t) = \frac{1}{\sin(t)} &\quad\sec(t) = \frac{1}{\cos(t)} &\quad\tan(t) = \frac{\sin(t)}{\cos(t)} &\quad\cot(t) = \frac{1}{\tan(t)} = \frac{\cos(t)}{\sin(t)}
\end{align*}
follow from definitions.
\item Pythagorean Identities:
\begin{align*}
\sin^2(t) + \cos^2(t) = 1 &\qquad\tan^2(t) + 1 = \sec^2(t) &\quad 1 + \cot^2(t) = \csc^2(t)
\end{align*}
Proofs: divide by $\cos^2(t)$ and $\sin^2(t)$ respectively. 
\end{enumerate}

\subsection{Finding all trig functions from the value of 1}
Given $\cos(t) = \frac{3}{5}$, and $t$ is in QIV.
{\color{blue}
\begin{align*}
\sin^2(t) + \frac{9}{25} = 1 \Rightarrow \sin^2(t) = \frac{16}{25} \Rightarrow \sin(t) &= - \frac{4}{5} \\
\tan(t) &= \frac{\sin(t)}{\cos(t)} = \frac{-4/5}{3/5} = -\frac{4}{3} \\
\csc(t) &= \frac{1}{\sin(t)} = -\frac{5}{4} \\
\sec(t) &= \frac{1}{\cos(t)} = \frac{5}{3} \\
\cot(t) &= \frac{1}{\tan(t)} = -\frac{3}{4} 
\end{align*}
}

\subsection{Writing one trig function in terms of another}
Write $\tan(t)$ in terms of $\cos(t)$ for $t$ in QIII.
We have
\begin{align*}
\tan(t) = \frac{\sin(t)}{\cos(t)}
\end{align*}
So we need to write $\sin(t)$ in terms of $\cos(t)$.
Using Pythagorean theorem,
\begin{align*}
\sin^2(t) = 1 - \cos^2(t) \Rightarrow \sin(t) = \pm \sqrt{ 1 - \cos^2(t) }
\end{align*}
we take the $-$ root since we're in QIII.

\end{document}