\documentclass{tufte-handout}

\title{4.4 Properties of Logarithms}

\author[AW]{Ammon Washburn}

\usepackage{graphicx} % allow embedded images
  \setkeys{Gin}{width=\linewidth,totalheight=\textheight,keepaspectratio}
  \graphicspath{{graphics/}} % set of paths to search for images
\usepackage{amsmath}  % extended mathematics
\usepackage{booktabs} % book-quality tables
\usepackage{units}    % non-stacked fractions and better unit spacing
\usepackage{multicol} % multiple column layout facilities
\usepackage{lipsum}   % filler text
\usepackage{enumerate}
\usepackage{wrapfig}
\usepackage{fancyvrb} % extended verbatim environments
  \fvset{fontsize=\normalsize}% default font size for fancy-verbatim environments
  \usepackage{tikz}

% Standardize command font styles and environments
\newcommand{\doccmd}[1]{\texttt{\textbackslash#1}}% command name -- adds backslash automatically
\newcommand{\docopt}[1]{\ensuremath{\langle}\textrm{\textit{#1}}\ensuremath{\rangle}}% optional command argument
\newcommand{\docarg}[1]{\textrm{\textit{#1}}}% (required) command argument
\newcommand{\docenv}[1]{\textsf{#1}}% environment name
\newcommand{\docpkg}[1]{\texttt{#1}}% package name
\newcommand{\doccls}[1]{\texttt{#1}}% document class name
\newcommand{\docclsopt}[1]{\texttt{#1}}% document class option name
\newenvironment{docspec}{\begin{quote}\noindent}{\end{quote}}% command specification environment

\newtheorem{mydef}{Definition}
\providecommand{\floor}[1]{\left \lfloor #1 \right \rfloor }

\begin{document}
\maketitle

\begin{abstract}
We will learn about properties of logarithms and practice them
\end{abstract}

\section{Laws of Logarithms}

\subsection{Laws}
\begin{enumerate}
\item product rule: $\log_a(AB) = \log_a(A) + \log_a(B)$ 
\item quotient rule: $\log_a\left( \frac{A}{B} \right) = \log_a(A) - \log_a(B)$ 
\item power rule: $\log_a(A^C) = C\log_a(A)$
\end{enumerate}

\subsection{Examples}
\begin{enumerate}
\item $\log_2(80) - \log_2(5) = log_2(\frac{80}{5}) = \log_2(16) = 4$
\item $\log_4(2) + 2\log_4(8) = \log_4(2) + \log_4(64) = \log_4(128) = 3.5$
\item $\log_2(6x) = \log_2(6) + \log_2(x)$
\item $\ln\left( \frac{ab}{\sqrt[3]{c}} \right) = \ln(ab) - \ln(c^{1/3}) = \ln(a) + \ln(b) - \frac{1}{3}\ln(c)$
\item $3\log(x) + \frac{1}{2}\log(x+1) = \log(x^3) + \log((x+1)^{1/2}) = \log(x^3\sqrt{x+1})$
\end{enumerate}

\subsection{NOT Examples}
\begin{enumerate}
\item $\log_a(x + y) \not= \log_a(x) + \log_a(y)$ (check with $a = x = y = 10$)
\item $\frac{\log_a(x)}{\log_a(y)} \not= \log_a(\frac{x}{y})$ (check with $a = x = y = 10$)
\item $\left( \log_a(x) \right)^y \not= y \log_a(x)$ (check with $a = x = y = 10$)
\end{enumerate}

\subsection{Problem 69: Wealth Distribution}
Vilfredo Pareto (1848 - 1923) observed that most of the wealth of a country is owned by a few members of the population. Pareto's Principle is
\begin{align*}
\log(P) = \log(c) - k\log(W)
\end{align*}
where $W$ is the wealth level (how much money a person has) and $P$ is the number of people in the population having that much money.
\begin{enumerate}[(a)]
\item Solve the equation for $P$.
\item Assume that $k = 2.1, c = 8000$, and $W$ is measured in millions of dollars. Use part (a) to find the number of people who have \$2 million or more. How many people have \$10 million or more?
\end{enumerate}
Answers:
\begin{enumerate}[(a)]
\item \begin{align*}
\log(P) &= \log(c) - \log(W^k) = \log\left(\frac{c}{W^k}\right) \Rightarrow
P = \frac{c}{W^k}
\end{align*}
\item Substituting in the values given, we have
\begin{align*}
P = \frac{8000}{(2)^{2.1}} \approx 1866.066; \qquad \qquad
P = \frac{8000}{(10)^{2.1}} \approx 63.546
\end{align*}
So according to the model, there are about 1866 people with \$2 million or more, and about 63 people with \$10 million or more.
\end{enumerate}

\subsection{Change of Base Formula}
Formula to change the base between logs:
\begin{align*}
\log_b(x) = \frac{\log_a(x)}{\log_a(b)}
\end{align*}

\subsection{Examples}
\begin{enumerate}
\item In calc, to find $\log_2(64) = \ln(64)/ln(2) = 6$ (as expected)
\item To graph, $\log_3(x + 2)$, set $y_1 = \log(x + 2)/\log(3)$ 
\end{enumerate}

\end{document}