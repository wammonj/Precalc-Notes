\documentclass{tufte-handout}

\title{Section 4.6: Modeling with Exponential and Logarithmic Functions}

\author[AW]{Ammon Washburn}

\usepackage{graphicx} % allow embedded images
  \setkeys{Gin}{width=\linewidth,totalheight=\textheight,keepaspectratio}
  \graphicspath{{graphics/}} % set of paths to search for images
\usepackage{amsmath}  % extended mathematics
\usepackage{booktabs} % book-quality tables
\usepackage{units}    % non-stacked fractions and better unit spacing
\usepackage{multicol} % multiple column layout facilities
\usepackage{lipsum}   % filler text
\usepackage{enumerate}
\usepackage{wrapfig}
\usepackage{fancyvrb} % extended verbatim environments
  \fvset{fontsize=\normalsize}% default font size for fancy-verbatim environments
  \usepackage{tikz}
  
  \usepackage{mathtools}
\usepackage{graphicx}
\usepackage{amssymb}
\usepackage{enumerate}
\usepackage{color}
\usepackage{fancyvrb}
\usepackage{breqn}
\usepackage{fancyhdr}
\usepackage{multicol}
%\usepackage[latin1]{inputenc}
\usepackage{tikz}
\usepackage{pgfplots}
\pgfplotsset{compat=1.8}

\definecolor{dkgreen}{rgb}{0,0.6,0}
\definecolor{gray}{rgb}{0.5,0.5,0.5}
\definecolor{mauve}{rgb}{0.58,0,0.82}

\newcommand{\R}[1]{\mathbb{R}^{#1}}

\pgfplotsset{vasymptote/.style={
    before end axis/.append code={
        \draw[densely dashed] ({rel axis cs:0,0} -| {axis cs:#1,0})
        -- ({rel axis cs:0,1} -| {axis cs:#1,0});
    }
}}
\pgfplotsset{hasymptote/.style={
    before end axis/.append code={
    	%\draw (axis cs:0,1) -- ({axis cs:0,1}-|{rel axis cs:1,0});
        \draw[densely dashed] ({rel axis cs:0,1} -| {axis cs:0,#1})
        -- ({rel axis cs:0,0} -| {axis cs:0,#1});
    }
}}

% Standardize command font styles and environments
\newcommand{\doccmd}[1]{\texttt{\textbackslash#1}}% command name -- adds backslash automatically
\newcommand{\docopt}[1]{\ensuremath{\langle}\textrm{\textit{#1}}\ensuremath{\rangle}}% optional command argument
\newcommand{\docarg}[1]{\textrm{\textit{#1}}}% (required) command argument
\newcommand{\docenv}[1]{\textsf{#1}}% environment name
\newcommand{\docpkg}[1]{\texttt{#1}}% package name
\newcommand{\doccls}[1]{\texttt{#1}}% document class name
\newcommand{\docclsopt}[1]{\texttt{#1}}% document class option name
\newenvironment{docspec}{\begin{quote}\noindent}{\end{quote}}% command specification environment

\newtheorem{mydef}{Definition}
\providecommand{\floor}[1]{\left \lfloor #1 \right \rfloor }

\begin{document}
\maketitle

\begin{abstract}
We will learn how to solve word problems with logarithmic and exponential models
\end{abstract}

%%%%%%%%%%%%%%%%%%%%%%%%%%%%%%%%%%%%%%%%%%%%%%%%%%%%%%%%%%%%%%%%%%%%%%%%%%
%% -------------------------- SECTION 1 ----------------------------------
%%%%%%%%%%%%%%%%%%%%%%%%%%%%%%%%%%%%%%%%%%%%%%%%%%%%%%%%%%%%%%%%%%%%%%%%%%

\section{Doubling Time}
Writing exponentials with base 2.
If the initial size of the population is $n_0$ and the doubling time is $a$, then the size of the population at time $t$ is
\begin{align*}
P(t) = P_0 2^{t/a}
\end{align*}
where $a$ and $t$ are measured in the same units (of time).

\subsection{Examples}
%\begin{enumerate}
%\item (\# 1) A culture of \textit{Streptococcus A} (bacterium) initially has 10 bacteria, and is observed to double every 1.5 hours.
%\begin{enumerate}
%\item Find an exponential model $n(t)$ for the number of bacteria in the culture after $t$ hours. \\
%{\color{blue} Given $a = 1.5$, $n_0 = 10$. So $n(t) = 10\cdot 2^{t/1.5}$.}
%\item Estimate the number of bacteria after 35 hours. \\
%{\color{blue} Given $t = 35$, so $n(35) = 10\cdot 2^{35/1.5} \approx 1.06 \cdot 10^8$.}
%\item When will the bacteria count reach 10,000? \\
%{\color{blue} Given $n = 10000$, so $10000 = 10 \cdot 2^{t/1.5} \Rightarrow 
%1000 = 2^{t/1.5} \Rightarrow 
%3 = (t/1.5)log(2) \Rightarrow
%t = \frac{4.5}{\log(2)} \approx 14.94868$.}
%\end{enumerate} 
%
%\item 
(\# 3) A grey squirrel population was introduced in a certain county of GB 30 years ago.
Biologists observe that the population doubles every 6 years, and now the population is 100,000.
\begin{enumerate}
\item What was the initial size of the squirrel population? \\
{\color{blue} We can write the equation $n(t) = 10^6 \cdot 2^{t/6}$ where $t$ is the years from present. 
So $30$ years ago, we have $n(-30) = 10^6 \cdot 2^{-30/6} = 10^6 \cdot 2^{-5} = 31,250$.}
\item Estimate the squirrel population 10 years from now. \\
{\color{blue} Given $t = 10$, so $n(10) = 10^6 \cdot 2^{10/6} \approx 3.1748\cdot 10^6$, or about 3.1748 million.}
\item Sketch a graph of the squirrel population.
\end{enumerate}

\subsection{Relative Growth Rate}
Writing exponentials with base $e$.
\textbf{Relative growth rate:} $r$, rate of population growth as a proportion of the population at any time $t$. \\
\textbf{Exponential growth:} $n(t) = n_0 e^{rt}$ (what do all these represent?)

\subsection{Example}
What is the relative growth rate of the squirrel population in prev. problem? \\

{\color{blue} We have $n(t) = n_0 \cdot 2^{t/6}$, want $n(t) = n_0 \cdot e^{rt}$. Setting these equal, and picking $t = 1$ (could pick any $t$ except 0), we have
\begin{align*}
2^{1/6} = e^r \Rightarrow r = \frac{1}{6}\ln(2)
\end{align*}
so $n(t) = n_0 \cdot e^{\ln(2)t/6}$.
This makes sense! $2 = e^{\ln(2)} \Leftrightarrow 2^{t/6} = \left( e^{\ln(2)} \right)^{t/6}$.}

(\# 1) A culture of \textit{Streptococcus A} (bacterium) initially has 10 bacteria, and is observed to double every 1.5 hours. How many bacteria will there be after 10 hours?\\

\subsection{Radioactive Decay}
Rate of decay proportional to mass.
Similar to population growth, but mass is decreasing.
In terms of \text{half-life}: how long it takes for 1/2 the mass to disappear.
Modeled by
\begin{align*}
m(t) = m_0 2^{-t/h} = m_0 \left(\frac{1}{2} \right)^{t/h}
\end{align*}
where $m_0$ is the initial mass, $h$ is the half-life, $t$ is time (same units as $h$).
Alternatively,
\begin{align*}
m(t) = m_0 e^{-rt} \qquad r = \frac{\ln(2)}{h}
\end{align*}
like we saw in the last example.

\subsection{Example}
(\# 17) The half-life of radium-226 is 1600 years. We have a 22-mg sample.
\begin{enumerate}[(a)]
\item Find half-life function.\\
{\color{blue} Given $m_0 = 22$, and $h = 1600$. So $m(t) = 22 \cdot 2^{-t/1600}$.}
\item Find exponential function.\\
{\color{blue} Using formula above, $m(t) = 22 \cdot e^{-\ln(2)t/1600}$.}
\item How much is left after 4000 years?\\
{\color{blue} Given $t = 4000$. So $m(4000) = 22 \cdot 2^{-4000/1600} = 22 \cdot 2^{-2.5} =  11/(2\sqrt{2}) \approx 3.889$ mg.}
\item How long until 18 mg left?\\
{\color{blue} $18 = 22 \cdot e^{-\ln(2)t/1600} \Rightarrow \frac{9}{11} = e^{-\ln(2)t/1600} \Rightarrow \ln\left( \frac{9}{11} \right) = -\ln{2}t/1600 \Rightarrow
t = -1600\ln\left( \frac{9}{11} \right)/\ln(2) \approx 463.21$ years.}
\end{enumerate}

\subsection{Newton's Law of Cooling}
Rate of cooling is proportional to temperature difference between object and surroundings.
Model:
\begin{align*}
T(t) = T_s + D_0 e^{-kt}
\end{align*}
where $T_s$ (surrounding temp), $D_0$ (initial temp difference), $k$ (positive const, depends on object), $t$ is in time, $T(t)$ is output temperature. 

\subsection{Example}
(\# 26) NLC is used in homicide investigations to determine the time of death.
The normal body temp is 98.6$^\circ$F.
Immediately following death, the body begins to cool.
It has been determined experimentally that the constant in NLC is approx. $k = 0.1947$, assuming that time is measured in hours.
Suppose the temp is 60$^\circ$F. 
\begin{enumerate}[(a)]
\item Find a function $T(t)$ that models the temperature $t$ hours after death. \\
{\color{blue} $T(t) = 60 + (98.6-60)e^{-0.1947t}$}
\item If the temperature of the body is now 72$^\circ$F, how long ago was the time of death? \\
{\color{blue} Given $T(t) = 72$, looking for $t$.
\begin{align*}
72 &= 60 + 38.6 e^{-0.1947t} \Rightarrow 12 = 38.6 e^{-0.1947t} \Rightarrow
-0.1947t = \ln\left( \frac{12}{38.6} \right) \Rightarrow \\
t &= \frac{-1}{0.1947} \ln \left( \frac{12}{38.6} \right)
\approx 6.0007
\end{align*}
so probably about 6 hours.}
\end{enumerate}

\end{document}