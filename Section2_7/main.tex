\documentclass{tufte-handout}

\title{Section 2.7 1-1 functions and their inverses}

\author[AW]{Ammon Washburn}

\usepackage{graphicx} % allow embedded images
  \setkeys{Gin}{width=\linewidth,totalheight=\textheight,keepaspectratio}
  \graphicspath{{graphics/}} % set of paths to search for images
\usepackage{amsmath}  % extended mathematics
\usepackage{booktabs} % book-quality tables
\usepackage{units}    % non-stacked fractions and better unit spacing
\usepackage{multicol} % multiple column layout facilities
\usepackage{lipsum}   % filler text
\usepackage[inline]{enumitem}
\usepackage{fancyvrb} % extended verbatim environments
  \fvset{fontsize=\normalsize}% default font size for fancy-verbatim environments

% Standardize command font styles and environments
\newcommand{\doccmd}[1]{\texttt{\textbackslash#1}}% command name -- adds backslash automatically
\newcommand{\docopt}[1]{\ensuremath{\langle}\textrm{\textit{#1}}\ensuremath{\rangle}}% optional command argument
\newcommand{\docarg}[1]{\textrm{\textit{#1}}}% (required) command argument
\newcommand{\docenv}[1]{\textsf{#1}}% environment name
\newcommand{\docpkg}[1]{\texttt{#1}}% package name
\newcommand{\doccls}[1]{\texttt{#1}}% document class name
\newcommand{\docclsopt}[1]{\texttt{#1}}% document class option name
\newenvironment{docspec}{\begin{quote}\noindent}{\end{quote}}% command specification environment

\newtheorem{mydef}{Definition}
\providecommand{\floor}[1]{\left \lfloor #1 \right \rfloor }

\begin{document}
\maketitle

\begin{abstract}
Learn when you can invert a function and what one-to-one means
\end{abstract}

\section{1-1 functions}
\begin{mydef}
A function f is one-to-one if $x \neq y$ then $f(x) \neq f(y)$.
\end{mydef}
Examples of function that are not one-to-one:
\begin{align}
f(x) & = x^2 & f(-2) & = f(2)\\
w(x) & = x(x-1)(x+3) & g(-3) & = g(1) \\
q(x) & = \pm \sqrt[]{x+2} & \textrm{Not even a} & \textrm{ function!}
\end{align}

\subsection{How to tell from a graph}
\begin{mydef}
If any horizontal line passes through the graph only once then the function is one-to-one.  This is called the horizontal line test.
\end{mydef}

\section{Inverse Functions}
\begin{mydef}
If f is one-to-one with domain A and range B then its inverse $f^{-1}$ has domain B and range A and is defined as follows:
\end{mydef}
\[f^{-1}(y) = x \Leftrightarrow f(x) = y\]

\noindent Examples:

\begin{table}
\centering
\begin{tabular}{c || c | c | c | c}
x & 1 & 2 & 3 & 4 \\
\hline
$g(x)$ & -1 & 2 & -3 & 5
\end{tabular}
\quad
\begin{tabular}{c || c | c | c | c}
x & -1 & 2 & -3 & 5 \\
\hline
$g^{-1}(x)$ & 1 & 2 & 3 & 4
\end{tabular}
\end{table}

Is g(x) one-to-one? Yes.  So we can do the inverse.  With tables you just flip the numbers and call it $g^{-1}(x)$.

\begin{mydef}
Function Inverse Property says that if $f$ and $f^{-1}$ are inverses of each other then $f \big ( f^{-1} (x) \big ) = x$ and $ f^{-1} \big ( f (x) \big ) = x$
\end{mydef}

\noindent Example: $r(x) = \frac{1}{x-5}$ and $e(x) = \frac{5x + 1}{x}$. $r \circ e (x) = x$ and $e \circ r (x) = x$

\subsection{How to find the inverse of an equation}

If given a function $f$ ($f(x) = \frac{x+1}{x+2}$) then do the following steps:
\begin{itemize}
\item Switch $y$ for $f(x)$ ~ ($y = \frac{x+1}{x+2}$)
\item Solve for x in terms of y ~ ($x = \frac{-2y + 1}{y -1}$)
\item Rewrite y as x and x as $f^{-1}(x)$ ($f^{-1}(x) = \frac{-2x + 1}{x -1}$)
\end{itemize}
It is important to not switch variables if they mean something. Like $P(h) = h^2$ where $h$ is the height and $P$ is the pressure.  These variables have meaning.

\section{Quiz}



\end{document}