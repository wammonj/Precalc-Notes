\documentclass{tufte-handout}

\title{4.2 and 4.3 Natural exponentials and logarithms}

\author[AW]{Ammon Washburn}

\usepackage{graphicx} % allow embedded images
  \setkeys{Gin}{width=\linewidth,totalheight=\textheight,keepaspectratio}
  \graphicspath{{graphics/}} % set of paths to search for images
\usepackage{amsmath}  % extended mathematics
\usepackage{booktabs} % book-quality tables
\usepackage{units}    % non-stacked fractions and better unit spacing
\usepackage{multicol} % multiple column layout facilities
\usepackage{lipsum}   % filler text
\usepackage{enumerate}
\usepackage{wrapfig}
\usepackage{fancyvrb} % extended verbatim environments
  \fvset{fontsize=\normalsize}% default font size for fancy-verbatim environments
  \usepackage{tikz}

% Standardize command font styles and environments
\newcommand{\doccmd}[1]{\texttt{\textbackslash#1}}% command name -- adds backslash automatically
\newcommand{\docopt}[1]{\ensuremath{\langle}\textrm{\textit{#1}}\ensuremath{\rangle}}% optional command argument
\newcommand{\docarg}[1]{\textrm{\textit{#1}}}% (required) command argument
\newcommand{\docenv}[1]{\textsf{#1}}% environment name
\newcommand{\docpkg}[1]{\texttt{#1}}% package name
\newcommand{\doccls}[1]{\texttt{#1}}% document class name
\newcommand{\docclsopt}[1]{\texttt{#1}}% document class option name
\newenvironment{docspec}{\begin{quote}\noindent}{\end{quote}}% command specification environment

\newtheorem{mydef}{Definition}
\providecommand{\floor}[1]{\left \lfloor #1 \right \rfloor }

\begin{document}
\maketitle

\begin{abstract}
We will talk about natural exponentials and logarithms
\end{abstract}

\section{Natural Exponential}
$e$ is a number between 2 and 3.  It comes about naturally in calculus problems and in modeling problems.

\subsection{Logistic Model and population growth models}

Consider the function $G(t) = \frac{C}{1+ke^{rt}}$.  This is a good model for things like the spread of disease and population growth.  Contrast it with exponential growth models. $G(t) = P e^rt$

Play with different values in the parameters to help them see what happens.

Let $n(x) = \frac{11200}{1+55e^{-0.044t}}$ represent the bird population in a habitat.  What was the initial population? Find the population after 10, 20, 30 years.  What value is it approaching?

\subsection{Continuous Interest}
If we have a rate that is compounded continuously then we model it with the following.

\[
A(t) = Pe^{rt}
\]

Example: How long will it take for your investment to double if your account has an interest rate of \%5.00 compounded continuously?

Do we need to know the principal amount?  No.  Just need $2=e^{.05t}$ and solve for $t$.  Wait how do we do this? We need the inverse of $e^x$.  The inverse of $e^x$ is $\ln(x)$.  So answer is $t = \frac{\ln (2)}{.05}$.  Plug in calculator.

\section{Logarithms}

Logarithms are the inverse functions of exponentials.  So $b^x = y$ iff $\log_{b} (y) = x$. 

Examples: Solve the following examples using these properties

\begin{multicols}{3}
\begin{enumerate}[(a)]
\item $e^{.05x} = 2$
\item $2^{x}= 1$
\item $(\frac{1}{2})^{2x}=4$
\item $\ln(x) = 3$
\item $\log_{10} (x) = 2$
\end{enumerate}
\end{multicols}

There are some properties of logarithms that you need to know that come about because they are inverses of exponentials.
\begin{multicols}{2}
\begin{itemize}
\item $\log_b (1) = 0$
\item $\log_b (b) = 1$
\item $\log_b (b^x) = x$
\item $b^{\log_b(x)}=x$
\end{itemize}
\end{multicols}

Examples:
\begin{multicols}{3}
\begin{enumerate}[(a)]
\item $\log(10^4)$
\item $\ln(e^4)$
\item $e^{\ln(5x)}$
\item $\log_2 (3 (\frac{1}{3} 2^4))$
\item $\log_2( \frac{1}{2})$
\item $\log_4(2^{2x})$
\end{enumerate}
\end{multicols}

Because they are inverses we already know their domain and range.  Domain: $(0,\infty)$ and Range: $(-\infty,\infty)$.  Since exponentials had a horizontal asymptote at $y=0$ then logarithms have a vertical asymptote at $x=0$.  What do the graphs look like?  Flipped exponentials.

Draw several graphs of logarithms for bases bigger and less than 1.

\subsection{Transformations on logarithms}

What happens to the domain if you do transformations in the horizontal directions? It changes the domain.  How to determine the domain of a scaled and shifted logarithmic function.

Domain of $h(x) = a \log_b(t(x)) + c$ is $t(x) > 0$.

Examples:

\begin{multicols}{2}
\begin{enumerate}[(a)]
\item $f(x) = 3 \ln(x+3) -2$
\item $e(x) = -2 \log_{10}(3(x-2)) +4$
\item $s(x) = -2\log_{10}(3x - 2) +4$
\item $y(x) = \log_2(-x-1)$
\end{enumerate}
\end{multicols}

What happens when you reflect the logarithm across an axis? Do some examples.

\end{document}