\documentclass{tufte-handout}

\title{Section 6.3 Trigonometric functions of angles}

\author[AW]{Ammon Washburn}

\usepackage{graphicx} % allow embedded images
  \setkeys{Gin}{width=\linewidth,totalheight=\textheight,keepaspectratio}
  \graphicspath{{graphics/}} % set of paths to search for images
\usepackage{amsmath}  % extended mathematics
\usepackage{booktabs} % book-quality tables
\usepackage{units}    % non-stacked fractions and better unit spacing
\usepackage{multicol} % multiple column layout facilities
\usepackage{lipsum}   % filler text
\usepackage{enumerate}
\usepackage{wrapfig}
\usepackage{fancyvrb} % extended verbatim environments
  \fvset{fontsize=\normalsize}% default font size for fancy-verbatim environments
\usepackage{tikz}
\usepackage{subcaption}
\captionsetup{compatibility=false}
\usepackage{mathtools}
\usepackage{graphicx}
\usepackage{amssymb}
\usepackage{enumerate}
\usepackage{color}
\usepackage{fancyvrb}
\usepackage{breqn}
\usepackage{fancyhdr}
\usepackage{multicol}
%\usepackage[latin1]{inputenc}
\usepackage{tikz}
\usepackage{pgfplots}
\pgfplotsset{compat=1.8}

\definecolor{dkgreen}{rgb}{0,0.6,0}
\definecolor{gray}{rgb}{0.5,0.5,0.5}
\definecolor{mauve}{rgb}{0.58,0,0.82}

\newcommand{\R}[1]{\mathbb{R}^{#1}}

\pgfplotsset{vasymptote/.style={
    before end axis/.append code={
        \draw[densely dashed] ({rel axis cs:0,0} -| {axis cs:#1,0})
        -- ({rel axis cs:0,1} -| {axis cs:#1,0});
    }
}}
\pgfplotsset{hasymptote/.style={
    before end axis/.append code={
    	%\draw (axis cs:0,1) -- ({axis cs:0,1}-|{rel axis cs:1,0});
        \draw[densely dashed] ({rel axis cs:0,1} -| {axis cs:0,#1})
        -- ({rel axis cs:0,0} -| {axis cs:0,#1});
    }
}}

% Standardize command font styles and environments
\newcommand{\doccmd}[1]{\texttt{\textbackslash#1}}% command name -- adds backslash automatically
\newcommand{\docopt}[1]{\ensuremath{\langle}\textrm{\textit{#1}}\ensuremath{\rangle}}% optional command argument
\newcommand{\docarg}[1]{\textrm{\textit{#1}}}% (required) command argument
\newcommand{\docenv}[1]{\textsf{#1}}% environment name
\newcommand{\docpkg}[1]{\texttt{#1}}% package name
\newcommand{\doccls}[1]{\texttt{#1}}% document class name
\newcommand{\docclsopt}[1]{\texttt{#1}}% document class option name
\newenvironment{docspec}{\begin{quote}\noindent}{\end{quote}}% command specification environment
\newcommand{\Z}[1]{\mathbb{Z}^{#1}}

\newtheorem{mydef}{Definition}
\providecommand{\floor}[1]{\left \lfloor #1 \right \rfloor }
\providecommand{\abs}[1]{| #1 |}


\begin{document}

\maketitle

\begin{abstract}
Apply our knowledge of right triangles to the 'unit circle' and get similar results
\end{abstract}

\begin{mydef}
Standard form of a right triangle given an angle $\theta$.  Place vertex of $\theta$ on the origin with the adjacent side on the x-axis.
\end{mydef}

In other words take a point $P(x,y)$ and we can make a right triangle out of that point.  Then we can define all the trig functions using $x$, $y$, and $r = \sqrt[]{x^2 + y^2}$.

\begin{multicols}{3}
\begin{itemize}
\item $\sin(\theta) = \frac{y}{r}$
\item $\cos(\theta) = \frac{x}{r}$
\item $\tan(\theta) = \frac{y}{x}$
\item $\cot(\theta) = \frac{x}{y}$
\item $\csc(\theta) = \frac{r}{y}$
\item $\sec(\theta) = \frac{r}{x}$
\end{itemize}
\end{multicols}

This is now exactly the same as the unit circle definition but we allow $r$ to be something other than one.

Note: The quadrantal angles are the ones that lie on the axis: $0$,$\frac{\pi}{2}$,$\pi$,$\frac{3pi}{2}$,$2\pi$ and all the others.

\section{Evaluating Trig functions}

Have class fill out table with trig function exact values for certain angles.

\begin{center}
\begin{tabular}{c | c | c | c | c | c | c}
Angle $\theta$ & $\sin(\theta)$ & $\csc(\theta)$ & $\cos(\theta)$ & $\sec(\theta)$ & $\tan(\theta)$ & $\cot(\theta)$ \\
\hline
0 & 0 & undefined & 1 & 1 & 0 & undefined \\
$\frac{\pi}{6}$ & $\frac{1}{2}$ & 2 & $\frac{\sqrt[]{3}}{2}$ & $\frac{2}{\sqrt[]{3}}$ & $ \sqrt[]{3}$ & $\frac{1}{\sqrt[]{3}}$ \\
$\frac{\pi}{4}$ & $\frac{\sqrt[]{2}}{2}$ & $\sqrt[]{2}$ & $\frac{\sqrt[]{2}}{2}$ & $\sqrt[]{2}$ & 1 & 1 \\
$\frac{\pi}{3}$ & $\frac{\sqrt[]{3}}{2}$ & $\frac{2}{\sqrt[]{3}}$ & $\frac{1}{2}$ & 2 & $\frac{1}{\sqrt[]{3}}$ & $ \sqrt[]{3}$ \\
$\frac{\pi}{2}$ & 1 & 1 & 0 & undefined & undefined & 0
\end{tabular}
\end{center}

All you need to know is the reference angle and the quadrant.  Won't talk about that here.

\section{Trig Identities}

We already know the following trig identities.  Use them in the examples.

\begin{align*}
\sin^2(x) + \cos^2(x) = 1 \\
\tan^2(x) + 1 = \sec^2(x) \\
1 + \cot^2(x) = \csc^2(x)
\end{align*}

\subsection{Examples}
\begin{enumerate}
\item Write $\tan(x)$ in terms of $\sec(x)$ where $x$ is in the III quadrant
\item Simplify $\tan^2(x) \cos^2(x) + \cos^2(x)$
\item Write $\cos(x)$ in terms of $\sin(x)$ where $x$ is in the II or III quadrant
\item If $\cot(\theta) = B$ and the terminal side is in quadrant II
\end{enumerate}

\section{Examples for section}
Try problems 67 and 71 in the textbook.
\end{document}