\documentclass{tufte-handout}

\title{Section 6.1 Angle measure}

\author[AW]{Ammon Washburn}

\usepackage{graphicx} % allow embedded images
  \setkeys{Gin}{width=\linewidth,totalheight=\textheight,keepaspectratio}
  \graphicspath{{graphics/}} % set of paths to search for images
\usepackage{amsmath}  % extended mathematics
\usepackage{booktabs} % book-quality tables
\usepackage{units}    % non-stacked fractions and better unit spacing
\usepackage{multicol} % multiple column layout facilities
\usepackage{lipsum}   % filler text
\usepackage{enumerate}
\usepackage{wrapfig}
\usepackage{fancyvrb} % extended verbatim environments
  \fvset{fontsize=\normalsize}% default font size for fancy-verbatim environments
\usepackage{tikz}
\usepackage{subcaption}
\captionsetup{compatibility=false}
\usepackage{mathtools}
\usepackage{graphicx}
\usepackage{amssymb}
\usepackage{enumerate}
\usepackage{color}
\usepackage{fancyvrb}
\usepackage{breqn}
\usepackage{fancyhdr}
\usepackage{multicol}
%\usepackage[latin1]{inputenc}
\usepackage{tikz}
\usepackage{pgfplots}
\pgfplotsset{compat=1.8}

\definecolor{dkgreen}{rgb}{0,0.6,0}
\definecolor{gray}{rgb}{0.5,0.5,0.5}
\definecolor{mauve}{rgb}{0.58,0,0.82}

\newcommand{\R}[1]{\mathbb{R}^{#1}}

\pgfplotsset{vasymptote/.style={
    before end axis/.append code={
        \draw[densely dashed] ({rel axis cs:0,0} -| {axis cs:#1,0})
        -- ({rel axis cs:0,1} -| {axis cs:#1,0});
    }
}}
\pgfplotsset{hasymptote/.style={
    before end axis/.append code={
    	%\draw (axis cs:0,1) -- ({axis cs:0,1}-|{rel axis cs:1,0});
        \draw[densely dashed] ({rel axis cs:0,1} -| {axis cs:0,#1})
        -- ({rel axis cs:0,0} -| {axis cs:0,#1});
    }
}}

% Standardize command font styles and environments
\newcommand{\doccmd}[1]{\texttt{\textbackslash#1}}% command name -- adds backslash automatically
\newcommand{\docopt}[1]{\ensuremath{\langle}\textrm{\textit{#1}}\ensuremath{\rangle}}% optional command argument
\newcommand{\docarg}[1]{\textrm{\textit{#1}}}% (required) command argument
\newcommand{\docenv}[1]{\textsf{#1}}% environment name
\newcommand{\docpkg}[1]{\texttt{#1}}% package name
\newcommand{\doccls}[1]{\texttt{#1}}% document class name
\newcommand{\docclsopt}[1]{\texttt{#1}}% document class option name
\newenvironment{docspec}{\begin{quote}\noindent}{\end{quote}}% command specification environment
\newcommand{\Z}[1]{\mathbb{Z}^{#1}}

\newtheorem{mydef}{Definition}
\providecommand{\floor}[1]{\left \lfloor #1 \right \rfloor }
\providecommand{\abs}[1]{| #1 |}


\begin{document}

\maketitle

\begin{abstract}
Learn about the basic properties of angles
\end{abstract}


\begin{mydef}
An angle $AOB$ consists of rays R1 and R2 with a common vertex $O$.  Interpet the angle as the rotation of R1 (intial side) onto R2 (terminal side).  As with a unit circle, counterclockwise rotation is positive.  The measure of an angle $\theta$ is the amount of rotation required.  How much it opens.
\end{mydef}

One unit of measurement of $\theta$ is the degree while another is radians.  Note if degree or rad is not given then assume radians.

\section{Conversion from radians to degrees}

\begin{itemize}
\item If you want to convert from radians to degrees then multiply by $\frac{180}{\pi}$
\item If you want to convert from degrees to radians then multiply by $\frac{\pi}{180}$
\end{itemize}

\subsection{Examples}
\begin{enumerate}[(a)]
\item Convert 120 degrees to radians
\item Convert $-\frac{\pi}{3}$ radians to degrees
\end{enumerate}

\section{Angles in Standard Position}

\begin{mydef}
An angle is in standard position if it is drawn in the xy-plane with vertex at the origin and the initial side on the x-axis (like last chapter).  Two angles are coterminal if terminal sides coincide.
\end{mydef}

\subsection{Examples}
\begin{enumerate}[(a)]
\item Angles coterminal to 2
\item Angles coterminal to $25^{\circ}$
\end{enumerate}

\section{Length of an Arc}

The length $s$ you travel around a circle of radius $r$ where you go an angle of $\theta$ is $s = r\theta$.

\subsection{Examples}
Do some examples in the textbook (Questions 51-53 are good enough)

\section{Area of Circular Sector}
Area of a circle: $A = \pi r^2$.  Area of a sector: $A = \frac{\theta}{2\pi}\pi r^2 = \frac{1}{2}\theta r^2$

\subsection{Examples}
Examples in the book (Questions 62-63 are good enough)

\section{Circular Motion}
A point moves along a circle of radius $r$ traverses $\theta$ radians in time $t$.  Travels distance $s=r\theta$.

There are two ways to talk about speed that are equivalent.  Angular speed: $\omega= \frac{\theta}{t}$ and Linear Speed: $v=\frac{s}{t}$.

Note that $v = \frac{s}{t} = \frac{r\theta}{t} = r \omega$.

\subsection{Examples}
Do example 64 in textbook and any others. 
\end{document}