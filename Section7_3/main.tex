\documentclass{tufte-handout}

\title{Section 7.3 Double Angle Formulas}

\author[AW]{Ammon Washburn}

\usepackage{graphicx} % allow embedded images
  \setkeys{Gin}{width=\linewidth,totalheight=\textheight,keepaspectratio}
  \graphicspath{{graphics/}} % set of paths to search for images
\usepackage{amsmath}  % extended mathematics
\usepackage{booktabs} % book-quality tables
\usepackage{units}    % non-stacked fractions and better unit spacing
\usepackage{multicol} % multiple column layout facilities
\usepackage{lipsum}   % filler text
\usepackage{enumerate}
\usepackage{wrapfig}
\usepackage{fancyvrb} % extended verbatim environments
  \fvset{fontsize=\normalsize}% default font size for fancy-verbatim environments
\usepackage{tikz}
\usepackage{subcaption}
\captionsetup{compatibility=false}
\usepackage{mathtools}
\usepackage{graphicx}
\usepackage{amssymb}
\usepackage{enumerate}
\usepackage{color}
\usepackage{fancyvrb}
\usepackage{breqn}
\usepackage{fancyhdr}
\usepackage{multicol}
%\usepackage[latin1]{inputenc}
\usepackage{tikz}
\usepackage{pgfplots}
\pgfplotsset{compat=1.8}

\definecolor{dkgreen}{rgb}{0,0.6,0}
\definecolor{gray}{rgb}{0.5,0.5,0.5}
\definecolor{mauve}{rgb}{0.58,0,0.82}

\newcommand{\R}[1]{\mathbb{R}^{#1}}

\pgfplotsset{vasymptote/.style={
    before end axis/.append code={
        \draw[densely dashed] ({rel axis cs:0,0} -| {axis cs:#1,0})
        -- ({rel axis cs:0,1} -| {axis cs:#1,0});
    }
}}
\pgfplotsset{hasymptote/.style={
    before end axis/.append code={
    	%\draw (axis cs:0,1) -- ({axis cs:0,1}-|{rel axis cs:1,0});
        \draw[densely dashed] ({rel axis cs:0,1} -| {axis cs:0,#1})
        -- ({rel axis cs:0,0} -| {axis cs:0,#1});
    }
}}

% Standardize command font styles and environments
\newcommand{\doccmd}[1]{\texttt{\textbackslash#1}}% command name -- adds backslash automatically
\newcommand{\docopt}[1]{\ensuremath{\langle}\textrm{\textit{#1}}\ensuremath{\rangle}}% optional command argument
\newcommand{\docarg}[1]{\textrm{\textit{#1}}}% (required) command argument
\newcommand{\docenv}[1]{\textsf{#1}}% environment name
\newcommand{\docpkg}[1]{\texttt{#1}}% package name
\newcommand{\doccls}[1]{\texttt{#1}}% document class name
\newcommand{\docclsopt}[1]{\texttt{#1}}% document class option name
\newenvironment{docspec}{\begin{quote}\noindent}{\end{quote}}% command specification environment
\newcommand{\Z}[1]{\mathbb{Z}^{#1}}

\newtheorem{mydef}{Definition}
\providecommand{\floor}[1]{\left \lfloor #1 \right \rfloor }
\providecommand{\abs}[1]{| #1 |}


\begin{document}

\maketitle

\begin{abstract}
Apply Addition formula to get double angle formula and learn how to use it
\end{abstract}

Using last section we know that
\begin{align*}
\sin(2x) = \sin(x+x) = \sin(x)\cos(x) + \sin(x)\cos(x) & = 2\sin(x)\cos(x) \\
\cos(2x) = \cos(x+x) = \cos(x)\cos(x) - \sin(x)\sin(x) & = \cos^2(x) - \sin^2(x) \\
& = 1 - 2\sin^2(x) \\
&2\cos^2(x) -1
\end{align*}

These are called the double angle formula.

\subsection{Examples}
\begin{enumerate}[(a)]
\item $\cos(x) = -\frac{2}{3}$ and $x$ is in quadrant II. Find $\cos(2x)+\sin(2x)$
\item Write $\cos(3x)$ in terms of $\cos(x)$
\end{enumerate}

\section{Formulas for Lowering powers}
We can rearragne the double sum formula for cosine and get the following equations:
\begin{align*}
\sin^2 & = \frac{1-\cos(2x)}{2} & \cos^2(x) & = \frac{1 + \cos(2x)}{2} & \tan^2(x) = \frac{1-\cos(2x)}{1+\cos(2x)}
\end{align*}
\subsection{Examples}
\begin{enumerate}
\item $\sin(x)^2\cos^2(x)$
\item $\cos(x)\sin^2(x)$
\item $\tan(x) = -\frac{4}{3}$, $x$ in quadrant II. Find $\sin(2x)$, $\cos(2x)$, and $\tan(2x)$
\item $\csc(x) = 4$, $\tan(x) < 0$. Find $\sin(2x)$, $\cos(2x)$, and $\tan(2x)$
\end{enumerate}

\end{document}